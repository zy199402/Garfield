\documentclass[10pt,letterpaper]{article}
\usepackage[T1]{fontenc}
\usepackage[parfill]{parskip} % line break instead of indentation
\usepackage[margin=1in]{geometry}
\usepackage{float}  % use attribute [H] to force images to stay where they should be
\usepackage{secdot} % dot after section number
\usepackage{lmodern}
\usepackage{amssymb,amsmath,amsthm,tikz-cd}
\usepackage{mathtools}
\usepackage{empheq}
\usepackage{enumerate}
\usepackage{nicefrac}
\usepackage{graphicx}
\usepackage{color}
\usepackage{hyperref}
\usepackage{tikz}
\newcommand{\N}{\mathbb{N}}
\newcommand{\Z}{\mathbb{Z}}
\newcommand{\Q}{\mathbb{Q}}
\newcommand{\R}{\mathbb{R}}
\newcommand{\C}{\mathbb{C}}

\renewcommand{\O}{\mathcal{O}} 

\makeatletter
\newcommand\course[1]{\renewcommand\@course{#1}}
\newcommand\@course{}
\newcommand\term[1]{\renewcommand\@term{#1}}
\newcommand\@term{}

\renewcommand{\@maketitle}{%
  {\bfseries \@course, \@term \hfill \@author}\
  \begin{center}
    {\Large \textbf \@title}
  \end{center}
}
\makeatother

\DeclarePairedDelimiter\ceil{\lceil}{\rceil}
\DeclarePairedDelimiter\floor{\lfloor}{\rfloor}

% use solid black square instead of box for QED
\renewcommand\qedsymbol{$\blacksquare$}


% bold title, italic text, extra space above and below;
\theoremstyle{plain}
\newtheorem{thm}{Theorem}[section]
\newtheorem{lem}[thm]{Lemma}
\newtheorem{prop}[thm]{Proposition}
\newtheorem*{cor}{Corollary}

% bold title, upright text, extra space above and below;
\theoremstyle{definition}
\newtheorem{defn}{Definition}[section]
\newtheorem{exmp}{Example}[section]
\newtheorem{xca}[exmp]{Exercise}

% italic title, upright text, no extra space above or below.
\theoremstyle{remark}
\newtheorem*{remark}{Remark}
\newtheorem*{claim}{Claim}
\newtheorem*{note}{Note}
\newtheorem{case}{Case}


% ================================================================================
\begin{document}
\course{Representations of Lie groups}
\term{}
\title{A criterion for semi-simple Lie algebra}
\author{Yu Zhao}
\maketitle

For a semisimple Lie algebra $\mathfrak{g}$ over a field $K$ of chracteristic $0$, we know that for all finite $\mathfrak{g}$ module , we have $H^{1}(\mathfrak{g}, A)=0$. And now we are going to prove the converse of this result is also true:

\begin{thm} For a Lie algebra $\mathfrak{g}$, if for all finite $\mathfrak{g}$-module $A$, we have that $H^{1}(\mathfrak{g}, A)=0$(*),then $\mathfrak{g}$ is semi-simple.
\end{thm}


Instead, we prove that if (*) condition holds for $\mathfrak{g}$, then for any exact sequence of $\mathfrak{g}$-module $0\to A' \to A \to A''\to 0$ will split. Thus we can decompose $\mathfrak{g}$ itself into simple Lie algebras with the representation $ad$.

For $\mathfrak{g}-$module $A,B$, consider $Hom_{K}(A,B)$ with the $\mathfrak{g}$-module structure $(g\phi)(a)=g(\phi(a))-\phi(ga)$. Then $Hom_{\mathfrak{g}}(A,B)=\{a\in Hom_{K}(A,B)| ga=0 \ , \forall g\}=Hom_{\mathfrak{g}}(K,Hom_{K}(A,B))=H^{0}(\mathfrak{g},Hom_{K}(A,B))$

Notice that $0\to Hom_{K}(A'',A')\to Hom_{K}(A,A') \to Hom_{K}(A,A')\to 0$. So we have the following exact sequence:


$0\to H^{0}(\mathfrak{g},Hom_{K}(A'',A'))\to H^{0}(\mathfrak{g},Hom_{K}(A,A'))  \to H^{0}(\mathfrak{g},Hom_{K}(A',A'))\to 0$. 

Since $H^{1}(\mathfrak{g}, Hom_{K}(A'',A'))=0$. Thus $Hom_{\mathfrak{g}}(A,A')\to Hom_{\mathfrak{g}}(A',A')$ is surjective, consdier one element of  $Hom_{\mathfrak{g}}(A,A')$ which maps to identity of $A'$, then it forms a split of the exact sequence.

The first extension of this theorem is that we can consider the cohomology of simple $\mathfrak{g}$-modules instead of all $\mathfrak{g}$-modules. For a module $A$ which is not simple, we can  consider a submodule $A'$ of $A$ and th exact sequence $0\to A' \to A \to A/A'$. Then we have $H^{1}(\mathfrak{g}, A')\to H^{1}(\mathfrak{g}, A) \to H^{1}(\mathfrak{g}, A/A')$. Thus $H^{1}(\mathfrak{g}, A)=0$ if $H^{1}(\mathfrak{g}, A')=0$ and $ H^{1}(\mathfrak{g}, A/A')=0$. With the induction on the dimension, we have the following result:

\begin{lem}
  If for any maximal ideal $\mathfrak{h}$ of $\mathfrak{g}$, we have $H^{1}(\mathfrak{g},\mathfrak{g}/\mathfrak{h})=0$, then for all finite modules $A$, we have $H^{1}(\mathfrak{g},A)=0$
\end{lem}

\begin{cor}
   For a Lie algebra $\mathfrak{g}$, if for all maximal ideals  $\mathfrak{h}$, we have that $H^{1}(\mathfrak{g}, \mathfrak{g}/\mathfrak{h})=0$, then $\mathfrak{g}$ is semi-simple.
\end{cor}


\end{document}


%%% Local Variables:
%%% mode: latex
%%% TeX-master: t
%%% End:
