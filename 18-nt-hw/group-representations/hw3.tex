\documentclass[10pt,letterpaper]{article}
\usepackage[T1]{fontenc}
\usepackage[parfill]{parskip} % line break instead of indentation
\usepackage[margin=1in]{geometry}
\usepackage{float}  % use attribute [H] to force images to stay where they should be
\usepackage{secdot} % dot after section number
\usepackage{lmodern}
\usepackage{amssymb,amsmath,amsthm}
\usepackage{mathtools}
\usepackage{empheq}
\usepackage{enumerate}
\usepackage{nicefrac}
\usepackage{graphicx}
\usepackage{color}
\usepackage{hyperref}
\usepackage{tikz}
\newcommand{\N}{\mathbb{N}}
\newcommand{\Z}{\mathbb{Z}}
\newcommand{\Q}{\mathbb{Q}}
\newcommand{\R}{\mathbb{R}}
\newcommand{\C}{\mathbb{C}}

\renewcommand{\O}{\mathcal{O}} 

\makeatletter
\newcommand\course[1]{\renewcommand\@course{#1}}
\newcommand\@course{}
\newcommand\term[1]{\renewcommand\@term{#1}}
\newcommand\@term{}

\renewcommand{\@maketitle}{%
  {\bfseries \@course, \@term \hfill \@author}\
  \begin{center}
    {\Large \textbf \@title}
  \end{center}
}
\makeatother

\DeclarePairedDelimiter\ceil{\lceil}{\rceil}
\DeclarePairedDelimiter\floor{\lfloor}{\rfloor}

% use solid black square instead of box for QED
\renewcommand\qedsymbol{$\blacksquare$}


% bold title, italic text, extra space above and below;
\theoremstyle{plain}
\newtheorem{thm}{Theorem}[section]
\newtheorem{lem}[thm]{Lemma}
\newtheorem{prop}[thm]{Proposition}
\newtheorem*{cor}{Corollary}

% bold title, upright text, extra space above and below;
\theoremstyle{definition}
\newtheorem{defn}{Definition}[section]
\newtheorem{exmp}{Example}[section]
\newtheorem{xca}[exmp]{Exercise}

% italic title, upright text, no extra space above or below.
\theoremstyle{remark}
\newtheorem*{remark}{Remark}
\newtheorem*{claim}{Claim}
\newtheorem*{note}{Note}
\newtheorem{case}{Case}


% ================================================================================
\begin{document}
\course{}
\term{}
\title{Homework  Solutions}
\author{}
\maketitle
\begin{itemize}
\item We define the product structure of group-like elements by the action $\nabla$, i.e. 

\begin{align*}
\cdot : G\times G \to  G \\
a \cdot b= \nabla (a \otimes b)
\end{align*}


We need to prove that this action is well defined, i.e. if $a$, $b$ are two  group like elements, then $a\cdot b$ is also group like. In fact $\Delta\nabla(a\otimes b)=(\nabla\otimes\nabla)(id\otimes \tau \otimes id)(\Delta \otimes \Delta) (a \otimes b)=(\nabla \otimes \nabla) (a\otimes b) \otimes (a \otimes b)$. And $\epsilon (1) (a\cdot b)= \epsilon(a)\cdot \epsilon(b)=1$. Hence $a\cdot b$ is also group like.

Then $\eta(1)$ is a unit in this action and this action is associative since $\nabla$ is a unit associative algebra. Besides, for $a \in G$, $\nabla (Sa \otimes a)= \eta\epsilon(a)=\eta(1) $. Hence $S$ form the inverse operation. Hence all group like elements form a group.

\item 3.3
$[a,[b,c]]= [[a,b],c]+ [c,[a,b]] \Leftrightarrow [a,[b,c]]-[[a,b],c]-[b,[a,c]]=0 \Leftrightarrow [a,[b,c]]+[c,[a,b]]+[b,[c,a]]=0 $ And through the jacobi identity we know that $[c,[a,b]]= [[[c,a],b]+[b,[c,a]]$ and hence $[[a,b],c]=-[b,[a,c]]+[b,[a,c]]$ which is equivalent to the fact that $ad[ab]=ad(a)ad(b)-ad(b)ad(a)$

\item 4.11
Notice that $dim V[k]$ is an additive fuction for the representations of $\mathfrak{sl}(2,\C)$. Hence $dim V[k]=\sum n_{j}dim V_{j}[k]=\sum_{j=0}n_{k+2j}$. Hence $dim V[k+2]-dim V[k]=n_{k} (*U)$. And by sum formula (*) over $k$, we got the other two formulas.

\end{itemize} 


\end{document}


%%% Local Variables:
%%% mode: latex
%%% TeX-master: t
%%% End:
