% Created 2016-03-09 Wed 11:57
\documentclass[11pt]{article}
\usepackage[utf8]{inputenc}
\usepackage[T1]{fontenc}
\usepackage{fixltx2e}
\usepackage{graphicx}
\usepackage{longtable}
\usepackage{float}
\usepackage{wrapfig}
\usepackage{rotating}
\usepackage[normalem]{ulem}
\usepackage{amsmath}
\usepackage{textcomp}
\usepackage{marvosym}
\usepackage{wasysym}
\usepackage{amssymb}
\usepackage{hyperref}
\tolerance=1000
\usepackage{fancyhdr}
\usepackage{amsmath,amsthm,amssymb}
\usepackage{tikz}
\author{Yu Zhao}
\date{\today}
\title{Homework}
\hypersetup{
  pdfkeywords={},
  pdfsubject={},
  pdfcreator={Emacs 24.5.1 (Org mode 8.2.10)}}
\begin{document}

\maketitle
\setcounter{secnumdepth}{-1}

\begin{enumerate}
\item Consider the map $f$ from $C$ to $C^{2}$ by $f(t)=(t^{3}, t^{5})$,
which is obviously a bijective map. In fact if $t_{1}\neq t_{2}$
but $f(t_{1})=f(t_{2})$, then we have $t_{1}$ and $t_{2}$ are all
nonzero, and induce that $t_{1}^{2}=t_{2}^{2}$ and hence
$t_{1}=t_{2}$. Similarly, this map is surjective. And this map is
continuous, hence we only need to check it is an open map.

For $t\neq 0$, $f$ is a local isomorphism, and for $t=0$, then
$f(B_{\epsilon})$ = $C \cap U(\epsilon^{3},\epsilon^{5})$, here $C$
is the algebraic curve $\{(x,y)|x^{5}=y^{3}\}$ and $U(a,b)
   =\{(x,y)| |x|<a, \ |y|<b\}$.

\item Consider the action $Z/pZ$ acts on $C^{2}$ by
$[1](z_{1},z_{2})=(e^{2\pi i/p}z_{1},e^{2\pi i q/p}z_{2})$.  with
$q, p$ coprime with $q$. Then the quotient of the action on $S^{3}$
is lens space $L(p,q)$. And if this quotient space is a manifold,
it must be dimension 4. Consider the neighborhood of $0$ into an
open subset of $R^{4}$, then we can induce a embedding of $L(p,q)$
in to $S^{4}$, which will not exist.

A example is that we can assume $p=2$ and $q=1$, then the lens space
is $P=RP^{3}$, if it can be embedded into $S^{4}$. We consider the
$Z_{2}$ cohomology. By Alexander's
duailty theorem, we have $\widetilde{H^{0}}(S^{4}-P)=Z_{2}^{2}$, thus
it has two connected component and we denote them $A$, $B$ to be
there closure. Then we have $A\cap B=P$ and $A\cup B=S^{4}$, by
Lefshetz duality and Mayer-Viectoris sequence we have
$H_{4}(A)\oplus H_{4}(B)=H^{4}(S^{4})/P= \tilde{H}^{0}(P)=0$. 

By the Mayer-Viectoris sequence again, we have
$H^{1}(A)+H^{1}(B)=Z_{2}$. We assume $H^{1}(A)=0$ and
$H^{1}(B)=Z_{2}$ generated by $b$. and hence we use the $Z/2Z$
cohomology, and thus found $b^{3}$ generate $H^{3}(P)$. Hence the
map from $H^{3}(P)$ to $H^{4}(S^{4})$ is a zero map. Thus we have
the following exact sequence: $Z_{2}=H^{4}(S^{4}) \simeq H^{4}(A)
   \oplus  H^{4}(B)$ contradiction!
\end{enumerate}
% Emacs 24.5.1 (Org mode 8.2.10)
\end{document}