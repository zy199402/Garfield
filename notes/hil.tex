% --------------------------------------------------------------
% This is all preamble stuff that you don't have to worry about.
% Head down to where it says "Start here"
% --------------------------------------------------------------

\documentclass[12pt]{article}

\usepackage[margin=1in]{geometry} 
\usepackage{amsmath,amsthm,amssymb,tikz-cd}
\newenvironment{definition}[2][Definition]{\begin{trivlist}
  \item[\hskip \labelsep {\bfseries #1}\hskip \labelsep {\bfseries #2.}]}{\end{trivlist}}
\newenvironment{theorem}[2][Theorem]{\begin{trivlist}
  \item[\hskip \labelsep {\bfseries #1}\hskip \labelsep {\bfseries #2.}]}{\end{trivlist}}
\newenvironment{lemma}[2][Lemma]{\begin{trivlist}
  \item[\hskip \labelsep {\bfseries #1}\hskip \labelsep {\bfseries #2.}]}{\end{trivlist}}
\newenvironment{exercise}[2][Exercise]{\begin{trivlist}
  \item[\hskip \labelsep {\bfseries #1}\hskip \labelsep {\bfseries #2.}]}{\end{trivlist}}
\newenvironment{reflection}[2][Reflection]{\begin{trivlist}
  \item[\hskip \labelsep {\bfseries #1}\hskip \labelsep {\bfseries #2.}]}{\end{trivlist}}
\newenvironment{proposition}[2][Proposition]{\begin{trivlist}
  \item[\hskip \labelsep {\bfseries #1}\hskip \labelsep {\bfseries #2.}]}{\end{trivlist}}
\newenvironment{corollary}[2][Corollary]{\begin{trivlist}
  \item[\hskip \labelsep {\bfseries #1}\hskip \labelsep {\bfseries #2.}]}{\end{trivlist}}

\begin{document}

% -------------------------------------------------------------- Start
% here --------------------------------------------------------------

% \renewcommand{\qedsymbol}{\filledbox}

\title{Hilbert Schemes and Quot schemes}%replace X with the appropriate number
\author{Yu Zhao} %if necessary, replace with your course title

\maketitle
The problem of constructing hilbert schemes is to modulize the closed subschemes of a projective varieies.
\section{The simplest case: when $X=P^{n}$ and the base scheme $S=Spec \ k$}
\label{sec:label}

In this case, first we need to scratch enough ``algebraic information'' of the closed subschemes or the quotient $\mathcal{O}_{X}$ ring. The first thing is the hilbert polynomial, which is an invariant information under flat morphisms. For fixed hilbert polynomial, the dimension of $H^0(X,\mathcal{F}(n))$ is invariant, hence we can consider the map from $H^0(X,\mathcal{O}(n))$ to $H^0(X,\mathcal{F}(n))$. Which is surjective for large $n$. And those infinite information is enough to verify all the closed subschemes, since for all $\mathcal{O}$ modules $F$, we have $F=\widetilde{\sum \Gamma(X, F(n))}$. Then those information are enough.

But we can only scratch finite ``critical'' information, and notice the module of $\mathcal{F}$ are finite generated, we only need to scratch the generators. And for the generality, we also need to gurantee that those generators have same orders, i.e. $H^0(X,\mathcal{F}(m+k))$ are all generated by $H^0(X,\mathcal{F}(m))$ for all $k \geq 0$ and all $\mathcal{F}$. Thus we need to gurantee the map:

\begin{displaymath}
  H^0(X,\mathcal{F}(m+l))\otimes H^0(X,\mathcal{O}(1))=  H^0(X,\mathcal{F}(m+l+1))
\end{displaymath}

A tool for this is Koszul complex: let $F=\sum_kH^0(X,\mathcal{F}(m+k))$, let $x_i$ are basis of $H^0(X,\mathcal{O}(1))$, then we can consider the Koszul complex by the action of $x_i$ on the module $F$. Hence our purpose is to check this Koszul complex is exact. But first we  notice that  this complex is exact in the sense of sheaves by direct caculation. Hence we can use the tool of spectral ssequence, by generate the cech complex on the other side. 

By caculating the spectral sequence through the vertical line, we have that it vanishes. By the vertical direction, and consider the right lower corner of the page, it reduces to the cohomology map:
\begin{displaymath}
  H^0(X,\mathcal{O}(1))\otimes H^p(X,\mathcal{F}(m)) \xrightarrow{d^1} H^0(X,\mathcal{F}(m+1))
\end{displaymath} 
has kernel 0. 
 

One way to gurantee is to make sure that for higher pages, the morphism on this element is 0. Hence we have the regularity condition.*


Thus we want to make sure that all subschemes with same hilbert polynomials will all have the regularity.

\section{Modificaton of this map to a base scheme}

Notice that, with a base change 
\section{Functorial property: why this map is a  locally closed immersion}
\label{sec:label}






\end{document}
